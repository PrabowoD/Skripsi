%!TEX root = ./template-skripsi.tex
%-------------------------------------------------------------------------------
% 								BAB I
% 							LATAR BELAKANG
%-------------------------------------------------------------------------------

\chapter{PENDAHULUAN}

\section{Latar Belakang Masalah}

% Intro
Indonesia merupakan sebuah negara yang memiliki berbagai macam 
kekayaan hayatinya, dan juga negara dengan jumlah pulau terbanyak di dunia. 
Salah satu dari kekayaan hayati yang dimiliki oleh Indonesia adalah ikan. Ikan 
merupakan bahan pangan yang paling banyak dikonsumsi oleh masyarakat 
Indonesia sejak zaman dahulu. Terbukti dengan banyaknya kuliner dengan bahan 
baku ikan di Indonesia. Selain menjadi bahan pangan masyarakat Indonesia juga 
memelihara ikan sebagai hiasan yang mereka taruh di dalam kolam ataupun 
akuarium.

Terdapat 2 cara untuk mendapatkan ikan yang pertama adalah menangkapnya pada habitatnya. 
Cara ini sangat efektif, cepat dan dapat dilakukan setiap hari, namun dengan menangkap 
pada habitatnya setiap hari akan mengakibatkan terjadi overfishing yang menyebabkan ikan 
berkurang dari habitat asli-nya. Cara kedua adalah membudidaya ikan. Budidaya ikan membutuhkan 
infrastruktur yang mendukung seperti lahan, kolam atau tambak yang memadai, bibit ikan yang akan 
dibudidayakan, dan serta pakan ikan dalam jumlah yang banyak. Walaupun mengeluarkan modal yang cukup besar 
dalam budidaya ikan tetapi budidaya ikan juga memberikan hasil yang sangat besar juga. Ikan hasil dari budidaya ikan dapat 
dijual hidup dari habitat aslinya, hal ini sangat karena banyak dari rumah makan atau restoran yang membutuhkan ikan hidup 
untuk menjaga kualitas dari makanan yang mereka sajikan. Ikan hidup juga mempunyai rasa yang berbeda dari ikan yang sudah 
mati pada saat dijual. Selain membudidayakan ikan sebagai bahan pangan, ada juga ikan yang dibudidayakan sebagai hiasan. 
Secara tidak langsung membudidayakan ikan telah mengurangi terjadinya overfishing. 
Pemerintah Indonesia mencatat pada tahun 2022 produksi ikan mencapai 17,76 juta ton dengan 5,57 juta ton dari budidaya ikan 
dan 5,97 juta ton dari hasil tangkap laut dan perairan umum daratan.

Pembudidayaan ikan di Indonesia sangat besar, tetapi memiliki masalah yang sama besarnya yaitu diperlukannya usaha 
untuk menghitung dan mengawasi jumlah ikan yang ingin dibudidayakan. Salah satu contohnya adalah penghitungan bibit ikan, 
dalam hal tersebut penghitungan bibit lele oleh para pedagang masih menggunakan cara manual (Al-Amri, 2020). 
Ikan dipindahkan kedalam sebuah wadah dan mengukur besar ikan dengan alat yang telah dibuat, lalu dipindahkan lagi satu persatu 
atau dihitung sesuai dengan berat untuk mendapatkan jumlah ikan. Cara-cara seperti tadi sangat tidak efisien dan atau tidak akurat, 
ikan yang diukur lalu dihitung satu persatu memerlukan waktu yang sangat lama bila memerlukan jumlah yang sangat banyak. 
Sementar teknik penimbangan hanya memberikan hasil perkiraan jumlah yang akurasinya masih perlu dipertanyakan. 

Masalah sangat penting dimana budidaya ikan sangat mementingkan banyak-nya jumlah ikan dalam tepat budidaya-nya, 
namun populasi ikan yang berlebihan akan memperlambat pertumbuhan ikan (Diansari et al, 2013), 
tapi disisi lain populasi ikan yang sangat sedikit akan mengurangi efisiensi lahan yang dimilik oleh peternak ikan. 
Dalam hal ini Al-Amri (2020) menciptakan sebuah sistem dapat menghitung jumlah ikan dengan menggunakan sensor proximity. 
Hasil dari uji coba yang didapat sangat baik dengan presentasi error rata-rata sebesar 4,07 dengan waktu selama 228 detik untuk 1000 bibit ikan, 
sedangkan jika dihitung secara manual akan membutuhkan waktu 20 menit untuk 1000 bibit ikan. 

Deteksi objek (Object detection) adalah salah satu dari visi komputer. Salah satu objektif nya adalah mengetahui lokasi dari sebuah objek pada gambar atau video. 
Pada penelitian Alim H (2021) telah membuat sebuah tracking movement ikan dengan menggunakan metode GMM dan Kalman filter penggunaan metode tersebut dapat memungkinkan 
pendeteksian dan mengamati pergerakan ikan, lalu pada penelitian Nugraha B (2022) pengekstraksi gambar menggunakan metode grabcut memudahkan mengekstrak gambar ikan dalam sebuah citra ikan. 
Namun kedua penelitian sebelumnya masih sangat general hanya untuk mendeteksi ikan saja dan masih belum memiliki fungsi lainnya. 

Pencocokan gambar adalah aspek dasar dari banyak permasalahan di dalam komputer, termasuk  pendeteksian benda atau pemandangan, memecahkan bangunan 3D dari banyak gambar, 
dan pelacakan Gerakan (D.G. Lowe, 2005). SIFT atau Scale Invariant feature transform (D.G. Lowe, 1999) pendekatan ini mengubah gambar menjadi kumpulan besar vektor fitur lokal, 
yang masing-masing tidak berubah terhadap terjemahan, penskalaan, dan rotasi gambar, dan sebagian tidak berubah terhadap perubahan iluminasi 
dan proyeksi affine atau 3D. SIFT sudah banyak digunakan untuk mencocokan gambar seperti Lokalisasi dan pemetaan dengan robot (Se, S et al, 2001), penyatuan panorama (Brown et al, 2003) dan lain-lain. 
SIFT di-identifikasi secara efisien dengan menggunakan pendekatan pemfilteran bertahap. Tahap pertama mengidentifikasi lokasi kunci dalam ruang skala dengan mencari lokasi yang maksimal atau minimal dari fungsi Different of Gaussian. 
Setiap titik digunakan untuk menghasilkan vektor fitur yang mendeskripsikan wilayah gambar lokal yang diambil sampelnya relatif terhadap bingkai koordinat ruang-skalanya. Fitur mencapai invarian parsial terhadap variasi lokal, 
seperti proyeksi affine atau 3D, dengan mengaburkan lokasi gradien gambar. Detection of local features invariant to affine transformations (K. Mikolajczyk, 2002) menjelaskan penggunaan dua kombinasi detektor yaitu Haris dan Laplacian-of-Gaussians. 
Kombinasi Haris-Laplace detector memberikan hasil yang lebih baik, Laplacian memungkinkan pemilihan skala karakteristik untuk titik yang diekstraksi dengan harris detector, dengan demikian descriptor dihitung pada lingkungan titik yang sama dalam gambar dengan resolusi yang berbeda, 
dan oleh karenanya tidak varian ke perubahan skala besar.
 
Berdasarkan latar yang telah dijelaskan, penulis mengusulkan untuk mengukur panjang serta menghitung berat ikan dengan menggunakan metode \emph{Harris-Laplace Detection}. 
Metode ini menggunakan dua kombinasi detektor yaitu Harris-detector dan Laplace-of-Gaussians. Hasil yang diharapkan adalah sebuah sistem yang mampu mendeteksi ikan pada gambar secara akurat dan menghitung beratnya.


\section{Rumusan Masalah}
Dari uraian pemasalahan diatas, perumusan masalah dalam penelitian ini adalah \textbf{“Bagaimana cara untuk mengukur panjang serta menghitung berat ikan menggunakan metode \emph{Harris-Laplace Detection}?”}

\section{Batasan Masalah}
\begin{enumerate}
	\item Penghitungan panjang dan berat dari pada ikan dengan menggunakan \emph{harris-laplace detection} sebagai pendeteksi benda
	\item penelitian dilakukan sampai mendapatkan hasil mendekati panjang dan berat dari ikan
	\item gambar-foto ikan yang akan digunakan ikan nila merah dan ikan lele
\end{enumerate}
	
\section{Tujuan Penelitian}
\begin{enumerate}
	\item Memahami arsitektur .
	\item Memahami cara kerja proses .
	\item Membuat implementasi .
\end{enumerate}

\section{Manfaat Penelitian}
\begin{enumerate}
	\item Bagi penulis
	

		
	\item Bagi Universitas Negeri Jakarta
	
	
			
\end{enumerate}

% Baris ini digunakan untuk membantu dalam melakukan sitasi
% Karena diapit dengan comment, maka baris ini akan diabaikan
% oleh compiler LaTeX.
\begin{comment}
\bibliography{daftar-pustaka}
\end{comment}
