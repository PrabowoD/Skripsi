%!TEX root = ./template-skripsi.tex
%-------------------------------------------------------------------------------
% 								BAB I
% 							LATAR BELAKANG
%-------------------------------------------------------------------------------

\chapter{PENDAHULUAN}

\section{Latar Belakang Masalah}

Indonesia merupakan negara kepulauan dengan kekayaan hayati yang melimpah, salah satunya adalah ikan yang menjadi sumber pangan utama masyarakat. Budidaya ikan di Indonesia terus berkembang untuk memenuhi kebutuhan konsumsi dan permintaan pasar, baik sebagai bahan pangan maupun ikan hias. Salah satu aspek penting dalam budidaya ikan adalah pengukuran panjang dan berat ikan, yang berperan dalam pemantauan pertumbuhan, penentuan dosis pakan, serta penentuan harga jual.

Namun, proses pengukuran panjang dan berat ikan di lapangan umumnya masih dilakukan secara manual, yaitu dengan menimbang dan mengukur satu per satu menggunakan alat ukur konvensional (\cite{Amri2020}). Cara ini tidak efisien, memakan waktu lama, dan berpotensi menimbulkan stres pada ikan sehingga dapat mempengaruhi kualitas dan pertumbuhan ikan. Selain itu, metode manual hanya memberikan estimasi jumlah atau berat secara keseluruhan, tanpa informasi detail mengenai distribusi ukuran ikan dalam satu populasi.

Seiring perkembangan teknologi, pengolahan citra digital menawarkan solusi otomatis untuk mengukur panjang dan berat ikan secara cepat dan akurat. Salah satu metode yang dapat digunakan adalah deteksi titik sudut (\emph{corner detection}) pada citra ikan. Titik sudut pada tubuh ikan, seperti ujung kepala dan ekor, dapat digunakan sebagai acuan untuk mengukur panjang ikan secara otomatis dari citra (\cite{Harris2013}). Dengan mengetahui panjang ikan, berat ikan juga dapat diestimasi menggunakan rumus atau model regresi yang sesuai (\cite{Diansari2013}).

Metode \emph{Harris-Corner Detection} merupakan salah satu algoritma deteksi sudut yang banyak digunakan karena kestabilannya terhadap rotasi, noise, dan efisiensi komputasi (\cite{Harris2013}). Dengan menerapkan metode ini pada citra ikan, sistem dapat secara otomatis mendeteksi titik-titik sudut penting, sehingga proses pengukuran panjang dan estimasi berat ikan dapat dilakukan secara digital, cepat, dan minim kontak langsung dengan ikan.

Berbagai penelitian sebelumnya telah mengembangkan metode untuk mendeteksi atau melacak ikan, seperti penggunaan metode GMM dan Kalman Filter untuk pelacakan gerakan (Alim, 2021), serta GrabCut untuk pemisahan objek dari latar belakang (Nugraha, 2022). Namun, metode-metode tersebut umumnya hanya fokus pada deteksi keberadaan ikan atau pelacakan pergerakan, dan belum secara langsung mengekstraksi fitur geometris seperti panjang dan berat ikan.

Di sisi lain, metode seperti SIFT (\emph{Scale Invariant Feature Transform}) (\cite{Lowe2004}) memang kuat terhadap perubahan skala dan rotasi, namun proses perhitungannya relatif kompleks dan memerlukan waktu komputasi lebih tinggi. Untuk kasus estimasi bentuk linear seperti panjang dan lebar, pendekatan berbasis deteksi sudut seperti \emph{Harris-Corner Detection} terbukti lebih efisien dan akurat dalam mendeteksi titik-titik sudut penting pada citra (\cite{Harris2013}). \emph{Harris-Corner Detection} memberikan kestabilan terhadap rotasi dan noise lokal serta memiliki struktur komputasi yang lebih ringan dibanding SIFT, sehingga cocok untuk diterapkan dalam sistem real-time atau perangkat keras terbatas.

Oleh karena itu, penelitian ini bertujuan untuk membangun sistem pengukuran panjang dan estimasi berat ikan berbasis citra digital menggunakan metode \emph{Harris-Corner Detection}. Diharapkan sistem ini dapat membantu pembudidaya ikan dalam melakukan monitoring pertumbuhan ikan secara efisien dan akurat, serta mendukung pengambilan keputusan dalam manajemen budidaya.

\section{Rumusan Masalah}
Berdasarkan pemaparan masalah di atas, perumusan masalah dalam penelitian ini adalah \textbf{“Bagaimana cara mengukur panjang serta menghitung berat ikan menggunakan metode \emph{Harris-Corner Detection}?”}

\section{Batasan Masalah}
\begin{enumerate}
	\item Sistem hanya menghitung panjang dan berat ikan dengan menggunakan \emph{Harris-Corners Detection}. 
	\item Jenis Ikan yang digunakan adalah ikan lele, ikan mas, dan ikan nila.
	\item Sumber gambar berupa dataset yang diambil langsung dari lapangan.
	\item Dataset telah dihilangkan latar belakangnya dan digantikan dengan warna solid hitam.
	\item Citra yang digunakan hanya citra tampak samping. 
	\item Bahasa Pemrograman menggunakan Python 3 atau lebih baru. 
\end{enumerate}
	
\section{Tujuan Penelitian}
	Tujuan dari Penelitian adalah Membangun sistem berbasis citra digital untuk mengestimasi panjang dan berat ikan menggunakan deteksi titik sudut dengan menggunakan metode \emph{Harris-corner detection}. 

\section{Manfaat Penelitian}
\begin{enumerate}
	\item Bagi penulis
	 Memperoleh gelar sarjana dalam bidang Ilmu Komputer, dan menambah pengalaman dalam pembangunan sebuah sistem komputer untuk aplikasi dunia nyata, serta pengetahuan tentang pendeteksian sudut atau \emph{corner} dari \emph{Harris-corner Detection}. 

		
	\item Bagi Program Studi Ilmu Komputer
	Penelitian "Penghitungan Panjang dan Berat Ikan Menggunakan Harris-Corners Detection" dapat dijadikan sebagai referensi dan menambah wawasan mahasiswa dan sivitas akademika Ilmu Komputer Universitas Negeri Jakarta.

\end{enumerate}

% Baris ini digunakan untuk membantu dalam melakukan sitasi
% Karena diapit dengan comment, maka baris ini akan diabaikan
% oleh compiler LaTeX.
\begin{comment}
\bibliography{daftar-pustaka}
\end{comment}
