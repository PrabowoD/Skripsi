%!TEX root = ./template-skripsi.tex
%-------------------------------------------------------------------------------
% 								BAB I
% 							LATAR BELAKANG
%-------------------------------------------------------------------------------

\chapter{PENDAHULUAN}

\section{Latar Belakang Masalah}

% Intro
Indonesia merupakan sebuah negara yang memiliki berbagai macam kekayaan hayatinya, dan juga negara dengan jumlah pulau terbanyak di dunia. 
Salah satu kekayaan hayati Indonesia adalah ikan. Ikan merupakan bahan pangan yang paling banyak dikonsumsi oleh masyarakat Indonesia sejak zaman dahulu. 
Terbukti dengan banyaknya kuliner dengan bahan baku ikan di Indonesia.

Terdapat beberapa cara untuk mendapatkan ikan yang pertama adalah menangkapnya pada habitatnya. 
Cara ini sangat efektif, cepat dan dapat dilakukan setiap hari, namun dengan menangkap pada habitatnya setiap hari akan mengakibatkan terjadi overfishing yang menyebabkan ikan berkurang dari habitat asli-nya. 
Cara kedua adalah membudidaya ikan. Budidaya ikan membutuhkan infrastruktur yang mendukung seperti lahan, kolam atau tambak yang memadai, bibit ikan yang akan dibudidayakan, dan serta pakan ikan dalam jumlah yang banyak. 
Walaupun mengeluarkan modal yang cukup besar, hasil dari budidaya ikan juga besar. 

Ikan hasil dari budidaya ikan dapat dijual hidup dari habitat aslinya, hal ini disebabkan oleh banyak dari rumah makan atau restoran yang membutuhkan ikan hidup untuk menjaga kualitas dari makanan yang mereka sajikan. 
Ikan hidup juga mempunyai rasa yang berbeda dari ikan yang sudah mati pada saat dijual. Selain dibudidayakan sebagai bahan pangan, ikan yang dibudidayakan sebagai ikan hias. 
Secara tidak langsung membudidayakan ikan telah mengurangi terjadinya overfishing. 
Pemerintah Indonesia mencatat pada tahun 2022 produksi ikan mencapai 17,76 juta ton dengan 5,57 juta ton dari budidaya ikan dan 5,97 juta ton dari hasil tangkap laut dan perairan umum daratan.

Pembudidayaan ikan di Indonesia sangat besar, tetapi memiliki masalah yang sama besarnya yaitu diperlukannya usaha untuk menghitung dan mengawasi jumlah ikan yang ingin dibudidayakan. 
Salah satu contohnya adalah penghitungan bibit ikan, dalam hal tersebut penghitungan bibit lele oleh para pedagang masih menggunakan cara manual (\cite{Amri2020}). 
Ikan dipindahkan kedalam sebuah wadah dan mengukur besar ikan dengan alat yang telah dibuat, lalu dipindahkan lagi satu persatu atau dihitung sesuai dengan berat untuk mendapatkan jumlah ikan. 
Cara-cara seperti tadi sangat tidak efisien dan atau tidak akurat, ikan yang diukur lalu dihitung satu persatu memerlukan waktu yang sangat lama bila memerlukan jumlah yang sangat banyak. 
Sementar teknik penimbangan hanya memberikan hasil perkiraan jumlah yang akurasinya masih perlu dipertanyakan. 

Masalah sangat penting dimana budidaya ikan sangat mementingkan banyak-nya jumlah ikan dalam tepat budidaya-nya, namun populasi ikan yang berlebihan akan memperlambat pertumbuhan ikan (\cite{Diansari2013}), 
tapi disisi lain populasi ikan yang sangat sedikit akan mengurangi efisiensi lahan yang dimilik oleh peternak ikan. 
Dalam hal ini (\cite{Amri2020}) menciptakan sebuah sistem dapat menghitung jumlah ikan dengan menggunakan sensor proximity. 
Hasil dari uji coba yang didapat sangat baik dengan presentasi error rata-rata sebesar 4,07 dengan waktu selama 228 detik untuk 1000 bibit ikan, 
sedangkan jika dihitung secara manual akan membutuhkan waktu 20 menit untuk 1000 bibit ikan. 

Deteksi objek (Object detection) adalah salah satu dari visi komputer. Salah satu objektif nya adalah mengetahui lokasi dari sebuah objek pada gambar atau video. 
Pada penelitian Alim H (2021) telah membuat sebuah tracking movement ikan dengan menggunakan metode GMM dan Kalman filter penggunaan metode tersebut dapat memungkinkan 
pendeteksian dan mengamati pergerakan ikan, lalu pada penelitian Nugraha B (2022) pengekstraksi gambar menggunakan metode grabcut memudahkan mengekstrak gambar ikan dalam sebuah citra ikan. 
Namun kedua penelitian sebelumnya masih sangat general hanya untuk mendeteksi ikan saja dan masih belum memiliki fungsi lainnya. 

Pencocokan gambar adalah aspek dasar dari banyak permasalahan di dalam komputer, termasuk pendeteksian benda atau pemandangan, memecahkan bangunan 3D dari banyak gambar, dan pelacakan Gerakan (\cite{Lowe2004}). 
SIFT atau \emph{Scale Invariant feature transform} (\cite{Lowe1999}) pendekatan ini mengubah gambar menjadi kumpulan besar vektor fitur lokal, yang masing-masing tidak berubah terhadap terjemahan, penskalaan, dan rotasi gambar, dan sebagian tidak berubah terhadap perubahan iluminasi dan proyeksi affine atau 3D. 
SIFT sudah banyak digunakan untuk mencocokan gambar seperti Lokalisasi dan pemetaan dengan robot, penyatuan panorama dan lain-lain. SIFT di-identifikasi secara efisien dengan menggunakan pendekatan pemfilteran bertahap. 

Tahap pertama mengidentifikasi lokasi kunci dalam ruang skala dengan mencari lokasi yang maksimal atau minimal dari fungsi \emph{Different of Gaussian}. 
Setiap titik digunakan untuk menghasilkan vektor fitur yang mendeskripsikan wilayah gambar lokal yang diambil sampelnya relatif terhadap bingkai koordinat ruang-skalanya. Fitur mencapai invarian parsial terhadap variasi lokal, 
seperti proyeksi affine atau 3D, dengan mengaburkan lokasi gradien gambar. \emph{Detection of local features invariant to affine transformations} (\cite{Mikolajczyk2004}) menjelaskan penggunaan detektor Haris. 
Kombinasi Haris detector memberikan hasil yang lebih baik, Laplacian memungkinkan pemilihan skala karakteristik untuk titik yang diekstraksi dengan \emph{Harris-Corner Detection}, dengan demikian descriptor dihitung pada lingkungan titik yang sama dalam gambar dengan resolusi yang berbeda, dan oleh karenanya tidak varian ke perubahan skala besar.

Berbagai metode telah dikembangkan untuk mendeteksi atau melacak ikan, seperti penggunaan metode GMM dan Kalman Filter yang efektif untuk pelacakan gerakan (Alim, 2021), atau GrabCut yang membantu dalam pemisahan objek dari latar belakang (Nugraha, 2022). 
Namun, metode-metode tersebut tidak dirancang untuk secara langsung mengekstraksi fitur geometris seperti panjang dan bentuk ikan.

Di sisi lain, metode seperti SIFT memang kuat terhadap skala dan rotasi, tetapi proses perhitungannya relatif kompleks dan memerlukan waktu komputasi lebih tinggi (\cite{Lowe2004}). 
Untuk kasus estimasi bentuk linear seperti panjang dan lebar, pendekatan berbasis deteksi sudut seperti \emph{Harris-Corner Detection} terbukti lebih efisien dan akurat dalam mendeteksi titik-titik sudut penting pada citra (\cite{Harris2013}). 
\emph{Harris-Corner Detection} memberikan kestabilan terhadap rotasi dan noise lokal serta memiliki struktur komputasi yang lebih ringan dibanding SIFT, sehingga cocok untuk diterapkan dalam sistem real-time atau dengan perangkat keras terbatas.

Oleh karena itu, dalam penelitian ini dipilih metode \emph{Harris-Corner Detection} untuk mengekstraksi sudut penting dari ikan dalam citra guna melakukan estimasi panjang dan berat secara otomatis. 
Hasil yang diharapkan adalah sebuah sistem yang mampu mengestimasi panjang serta berat ikan pada gambar secara akurat


\section{Rumusan Masalah}
Berdasarkan pemaparan masalah diatas, perumusan masalah dalam penelitian ini adalah \textbf{“Bagaimana cara mengukur panjang serta menghitung berat ikan menggunakan metode \emph{Harris-Corner Detection}?”}

\section{Batasan Masalah}
\begin{enumerate}
	\item Sistem hanya menghitung panjang dan berat ikan dengan menggunakan menggunakan \emph{harris-corners detection}. 
	\item Jenis Ikan yang digunakan adalah ikan lele, ikan mas, dan ikan nila.
	\item Sumber gambar berupa dataset yang diambil langsung dari lapangan dan telah dihilangkan latar belakangnya.
	\item Citra yang digunakan hanya tampak samping. 
	\item Bahasa Pemrograman menggunakan python 3 atau lebih. 
	\item 
\end{enumerate}
	
\section{Tujuan Penelitian}
	Tujuan dari Penelitian adalah Membangun sistem berbasis citra digital untuk mengestimasi panjang dan berat ikan menggunakan deteksi titik sudut dengan menggunakan metode \emph{Harris-corner detection}. 

\section{Manfaat Penelitian}
\begin{enumerate}
	\item Bagi penulis
	 Memperoleh gelar sarjana dalam bidang Ilmu Komputer, dan menambah pengalaman dalam pembangunan sebuah sistem operasi komputer dengan aplikasi dunia nyata, serta pengetahuan tentang pendeteksian sudut atau korner dari \emph{Harris-corner Detection}. 

		
	\item Bagi Program Studi Ilmu Komputer
	Penelitian "Penghitungan Panjang Dan Berat Ikan Menggunakan Harris-Corners Detection" bisa dapat dijadikan sebagai referensi dan menambah wawasan warga prodi Ilmu Komputer Universitas Negeri Jakarta.
		
\end{enumerate}

% Baris ini digunakan untuk membantu dalam melakukan sitasi
% Karena diapit dengan comment, maka baris ini akan diabaikan
% oleh compiler LaTeX.
\begin{comment}
\bibliography{daftar-pustaka}
\end{comment}
